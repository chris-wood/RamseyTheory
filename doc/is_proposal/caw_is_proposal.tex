\documentclass[paper=a4, fontsize=11pt]{scrartcl} % A4 paper and 11pt font size

\usepackage[T1]{fontenc} % Use 8-bit encoding that has 256 glyphs
\usepackage{fourier} % Use the Adobe Utopia font for the document - comment this line to return to the LaTeX default
\usepackage[english]{babel} % English language/hyphenation
\usepackage{amsmath,amsfonts,amsthm} % Math packages

\usepackage{lipsum} % Used for inserting dummy 'Lorem ipsum' text into the template

\usepackage{sectsty} % Allows customizing section commands
%\allsectionsfont{\normalfont\scshape} % Make all sections centered, the default font and small caps

\usepackage{fancyhdr} % Custom headers and footers
\pagestyle{fancyplain} % Makes all pages in the document conform to the custom headers and footers
\fancyhead{} % No page header - if you want one, create it in the same way as the footers below
\fancyfoot[L]{} % Empty left footer
\fancyfoot[C]{} % Empty center footer
\fancyfoot[R]{\thepage} % Page numbering for right footer
\renewcommand{\headrulewidth}{0pt} % Remove header underlines
\renewcommand{\footrulewidth}{0pt} % Remove footer underlines

\usepackage[compact]{titlesec}
\titlespacing{\section}{0pt}{*0}{*0}
\titlespacing{\subsection}{0pt}{*0}{*0}
\titlespacing{\subsubsection}{0pt}{*0}{*0}

\setlength{\headheight}{13.6pt} % Customize the height of the header
\setlength{\parskip}{\baselineskip}%
\setlength{\parsep}{0pt}
\setlength{\headsep}{0pt}
\setlength{\topskip}{0pt}
\setlength{\topmargin}{0pt}
\setlength{\topsep}{0pt}
\setlength{\partopsep}{0pt}

%\numberwithin{equation}{section} % Number equations within sections (i.e. 1.1, 1.2, 2.1, 2.2 instead of 1, 2, 3, 4)
%\numberwithin{figure}{section} % Number figures within sections (i.e. 1.1, 1.2, 2.1, 2.2 instead of 1, 2, 3, 4)
%\numberwithin{table}{section} % Number tables within sections (i.e. 1.1, 1.2, 2.1, 2.2 instead of 1, 2, 3, 4)

\setlength\parindent{0pt} % Removes all indentation from paragraphs - comment this line for an assignment with lots of text

%----------------------------------------------------------------------------------------
%	TITLE SECTION
%----------------------------------------------------------------------------------------

\newcommand{\horrule}[1]{\rule{\linewidth}{#1}} % Create horizontal rule command with 1 argument of height

\title{	
\normalfont \normalsize 
\textsc{Department of Computer Science, Rochester Institute of Technology} \\ [25pt] % Your university, school and/or department name(s)
\horrule{2pt} \\[0.4cm] % Thin top horizontal rule
\huge Study of the Edge Folkman Number Bounds \\ 
\Large Independent Study Proposal \\
\horrule{2pt} \\[0.5cm] % Thick bottom horizontal rule
}

\author{Christopher Wood \\ Advisor: Professor Stanis{\l}aw Radziszowski} % Your name

\date{\normalsize\today} % Today's date or a custom date

\begin{document}

\maketitle % Print the title

\section{Background}
Edge Folkman numbers, first introduced by Folkman in 1970 \cite{Folkman}, are concerned
with the graphs in which a monochromatic copy of a particular subgraph exists for all edge colorings. 
We write $G \to (a_1, ..., a_k; p)^e$ iff for every edge coloring of an undirected simple graph $G$ not 
containing $K_p$, there exists a monochromatic $K_{a_{i}}$ in color $i$ for some $i \in \{1, ..., k\}$. 
The edge Folkman number is defined as $F_e(a_1, ..., a_k) = \min\{|V(G)| : G \to (a_1, ..., a_k; p)^e\}$.
Similarly, the vertex Folkman number is defined as $F_v(a_1, ..., a_k) = \min\{|V(G)| : G \to (a_1, ..., a_k; p)^v\}$.
In 1970 Folkman proved that for all $k > \max(s,t)$, edge- and vertex- Folkman numbers $F_e(s,t;k$
and $F_v(s,t;k)$ exist. Prior to this, Erd\H{o}s and Hajnal posed the problem of finding $F_e(3,3;4)$, which
can be equivalently stated as the following question \cite{Erdos01}: 

\noindent \emph{What is the order of the smallest $K_4$-free graph for which any $2$-coloring of 
its edges must contain at least one monochromatic triangle?}

This is equivalent to finding the smallest $K_4$-free graph that is not the union of two 
triangle-free graphs. Since the proposition of this problem, there has been a significant
amount of work aimed at narrowing the upper and lower bounds of $F_e(3,3;4)$. 
Table \ref{tab:history} enumerates main developments of the previous work on this problem 
and leads us to the current state of the field.

\begin{table}
\caption{History of $F_e(e,e;4)$}
\begin{center}
	\begin{tabular}{c|c|l|c}
	\hline
	Year & Bounds & Who & Ref. \\ \hline
	1967 & any? & Erd\H os-Hajnal & \cite{Erdos01} \\
	1970 & exist & Folkman & \cite{Folkman} \\
	1972 & $\geq 10$ & Lin & \cite{lin} \\
	1975 & $\leq 10^{10}$? & Erd\H{o}s offers \$100 for proof & ~ \\
	1986 & $\leq 8 \times 10^{11}$ & Frankl-R\"{o}dl & \cite{frankl86} \\
	1988 & $\leq 3 \times 10^9$ & Spencer & \cite{spencer88} \\
	1999 & $\geq 16$ & Piwakowski et al. (implicit) & \cite{piwakowski99} \\
	2007 & $\geq 19$ & Radziszowski-Xu & \cite{spr07} \\
	2008 & $\leq 9697$ & Lu & \cite{lu08} \\
	2008 & $\leq 941$ & Dudek-R\"{o}dl & \cite{dudek08} \\
	2012 & $\leq 786$ & Lange et al. & \cite{arlFolkman} \\
	2012 & $\leq 100$? & Garaham offers \$100 for proof & ~ \\
	\hline
	\end{tabular}
\end{center}
\label{tab:history}
\end{table}

\section{Two Problems}
\subsection{Lower Bound}
The current lower bound for $F_e(3,3;4)$ stands at $19$ \cite{spr07}. The proof technique for this bound,
which relies on the fact that $H \to (3,3;4)^v \Rightarrow H + x \to (3,3;5)^e$, constructed all 
nonisomorphic graphs $G$ on $18$ vertices that contained independent sets $I$ of cardinality $4$. Then, for
each graph $G$, the authors used the fact that all induced subgraphs $H \in \mathcal{F}_v(3,3;4)$, where $H$ is 
composed of the vertex set $V(G) \setminus I$. Using computations, the authors then constructed
all possible candidate graphs $G$ from these subgraphs $H$ by taking the union of the vertices in
$I$ with every vertex set $A \subset M$, where $M$ is the set of all subsets $A \subset V(H)$ such that
the graph induced by $A$ is a maximal triangle-free subgraph, and testing to see whether the resultant
graph $G$ arrows $(3,3)^e$. The fact that $\chi(G) \geq 6$ if $G \in \mathcal{F}_e(3,3;4)$ was used to further
restrict the candidate graphs $G$ that were checked. 

\subsection{Upper Bound}
Dudek and R\"{o}dl proved that $G \to (3,3)^e$ if and only if $MC(H_G) < 2t_{\Delta}(G)$ \cite{dudek08},
where $MC(H_G)$ is the size of the maximum cut of graph $H = (E(G), \{(e_1, e_2) | \{e_1, e_2, e_3\}$ form a triangle
in $G\}$. Since MAX-CUT is an NP-complete problem, computations using this fact rely on approximation
algorithms, such as the one proposed by Geomans and Williamson \cite{geomans95-maxcut}. 
In fact, Lange et al. \cite{arlFolkman}
used this approximation approach, which formulates the MAX-CUT problem as a semidefinite program (SDP).
In their work, the authors examined graphs of the form 
$G(n, r) = (\mathbb{Z}_{n}, \{(x,y) | x \not= y, x - y \equiv \alpha^r \mod n, \alpha \in \mathbb{Z}_{n}\}$.
Using these computations, the authors were able to show that $MC(H_{G_{786}}) \leq 857750$, where
$G_{786} = G(786, 3)$, and since $2t_{\Delta}(G_{786}) = 857762$, it is clear that $F_e(3,3;4) \leq 786$. 

It has been conjectured that $F_e(3,3;4) \leq 127$, which is motivated by the graph 
$G_{127} = (\mathbb{Z}_{127}, E = \{(x,y) | x - y = \alpha^3 \mod 127\})$ \cite{spr07}.
The intuition for this conjecture is that $G_{127}$ has a large number of triangles and many small dense subgraphs.
Proving or disproving the conjecture that $G_{127} \to (3,3)^e$ would be a significant result for the upper bound of
$F_e(3,3;4)$.

\section{Summary of Proposed Study}
The ultimate goal of this work is to attack both the upper and lower bounds of $F_e(3,3;4)$. 
Unfortunately, it will not be feasible to leverage the same technique used by Radziszowski 
et al. to prove the current bound of $19$. The reason for this is that the estimated number of independent
sets of size $5$ in nonisomorphic graphs on $19$ vertices is expected to be more than $10^{19}$ \cite{spr1995}.
Clearly, the number of candidate graphs $G$ to check for $G \to (3,3)^e$ needs to be smaller, so we will
work towards devising constraints similar to the chromatic number $\chi(G) \geq 6$. We will also investigate
other facts about graph structure imposed by membership in the set $\mathcal{F}_e(3,3;4)$. 

Attacking the upper bound will be an entirely different form of computation. In particular, we will need to find a 
graph $G$ on $n$ vertices where $G \to (3,3)^e$. As previously discussed, we will be focusing on the graph 
$G_{127} = G(127,3)$ in an attempt to prove the conjecture that $G_{127} \to (3,3)^e$. 
To determine this we will decompose the problem of arrowing into problems on subgraphs $H$ that 
witness $H \not \to (3,3)^e$, and then carefully extend $H$ to encompass all of $G$. 

Due to the computationally intensive nature of this procedure, we will 
reduce $\{H | H \not \to (3,3)^e\}$ to 3-SAT and leverage
the power of SAT solvers such as Minisat \cite{minisat} and zChaff \cite{zchaff}. The reduction works by
mapping the edges in $E(H)$ to variables in $\phi_G \in$ 3-SAT, and for edge adding the following
clauses to $\phi_G$:
\begin{align*}
(x + y + z) \wedge (\bar{x} + \bar{y} + \bar{z})
\end{align*}
Clearly, $H \not \to (3,3)^e \Leftrightarrow \phi_G$ is satisfiable. Therefore, if $\phi_G$ is not satisfiable,
then $G \to (3,3)^e$, and so $F_e(3,3;4)$.

A major part of this work will be identifying candidate subgraphs $H$ for extension that 
can easily be solved using the aforementioned 3-SAT solvers. We will also take this opportunity
to compare the performance of popular k-SAT solvers published in the literature \cite{zchaff} \cite{minisat}.

\section{Project Goals and Deliverables}
The following project goals have been identified for this work:
\begin{itemize}
	\item Research current methods and potential techniques for computing the upper and lower bounds of Folkman 
	numbers (in particular, $F_e(3,3;4)$).
	\item Use SAT and SDP solvers in an attempt to solve the $G_{127} \to (3,3)^e$ conjecture.
	\item Implement software to aid in the computation of Folkman number bounds.
	\item Begin a survey paper outlining the history of Folkman number research \cite{sprSurvey}.
\end{itemize}

By achieving these goals I will generate the following deliverables:
\begin{itemize}
	\item Publication-ready paper and more in-depth progress report for the entire project.
	\item Report that weighs the performance of various SAT solvers for solving this arrowing problem with $G_{127}$.
	\item Report discussing the subgraph pruning techniques used for the lower bound.
\end{itemize}

%%%%% REFERENCES %%%%%

\begin{thebibliography}{9}

\bibitem{Folkman} Jon Folkman. Graphs with monochromatic complete subgraphs in
every edge coloring. \emph{SIAM Journal of Applied Mathematics}. 18 (1970), 19-24.

\bibitem{Erdos01} P. Erd\H{o}s, A. Hajnal. Research problem 2-5. \emph{Journal of Combinatory Theory}, 
2 (1967), 104.

\bibitem{lin} Shen Lin. On Ramsey numbers and $K_r$-coloring of graphs. \emph{Journal of
Combinatiorial Theory, Series B}, 12:82-92, 1972.

\bibitem{spr1995} Brendan D. McKay and Stanis{\l}aw P. Radziszowski. R(4,5) = 25. \emph{Journal
of Graph Theory}, 19(1995), 309-322.

\bibitem{frankl86} Peter Frankl and Vojtech R\"{o}dl. Large triangle-free subgraphs
in graphs with $K_4$. \emph{Graphs and Combinatorics}, 2:135-144, 1986.

\bibitem{spencer88} Joel Spencer. Three hundred million points suffice. \emph{Journal of
Combinatiorial Theory, Series A}, 49(2):210-217, 1988. Also see erratum by M. Hovey in
Vol. 50, p. 323.

\bibitem{piwakowski99} Konrad, Piwakoswki, Stanis{\l}aw P. Radziszowski, and Sebastian
Urba\'nski. Computation of the Folkman Number $F_e(3,3;5)$. \emph{Journal of Graph
Theory}, 32:41-49, 1999.

\bibitem{spr07} Stanis{\l}aw P. Radziszowski and Xiaodong Xu. On the Most Wanted
Folkman Graph. \emph{Geocombinatiorics}, 16(4):367-381, 2007.

\bibitem{lu08} Linyuan Lu. Explicit Construction of Small Folkman Graphs. \emph{SIAM Journal
on Discrete Mathematics}, 21(4):1053-1060, January 2008.

\bibitem{dudek08} Andrzej Dudek and Vojtech R\"{o}dl. On the Folkman Number $f(2,3,4)$.
\emph{Experimental Mathematics}, 17(1):63-67, 2008.

\bibitem{minisat} Niklas S\''{o}rensson and Niklas Een. Minisat v1.13 - A SAT Solver with Conflict-Clause Minimization. \emph{SAT 2005}, (2005): 53.

\bibitem{zchaff} Yogesh Mahajan, Fu Zhaohui, and Sharad Malik. Zchaff2004: An Efficient SAT Solver. \emph{Theory and Applications of Satisfiability Testing}. Springer Berlin/Heidelberg, 2005.

\bibitem{sprSurvey} Stanis{\l}aw P. Radziszowski. Small Ramsey Numbers. \emph{Electronic Journal of Combinatorics}, 1 (2011).

\bibitem{geomans95-maxcut} Michael Goemans and David Williamson. Improved Maximum Approximation Algorithms for Maximum Cut and Satisability Problems Using Semidefinite Programming. \emph{Journal of the ACM}, 42(6):1115-1145, 1995.

\bibitem{arlFolkman} TODO

\end{thebibliography}

\end{document}