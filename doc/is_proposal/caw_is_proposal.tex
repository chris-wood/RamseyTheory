\documentclass[paper=a4, fontsize=11pt]{scrartcl} % A4 paper and 11pt font size

\usepackage[T1]{fontenc} % Use 8-bit encoding that has 256 glyphs
\usepackage{fourier} % Use the Adobe Utopia font for the document - comment this line to return to the LaTeX default
\usepackage[english]{babel} % English language/hyphenation
\usepackage{amsmath,amsfonts,amsthm} % Math packages

\usepackage{lipsum} % Used for inserting dummy 'Lorem ipsum' text into the template

\usepackage{sectsty} % Allows customizing section commands
\allsectionsfont{\centering \normalfont\scshape} % Make all sections centered, the default font and small caps

\usepackage{fancyhdr} % Custom headers and footers
\pagestyle{fancyplain} % Makes all pages in the document conform to the custom headers and footers
\fancyhead{} % No page header - if you want one, create it in the same way as the footers below
\fancyfoot[L]{} % Empty left footer
\fancyfoot[C]{} % Empty center footer
\fancyfoot[R]{\thepage} % Page numbering for right footer
\renewcommand{\headrulewidth}{0pt} % Remove header underlines
\renewcommand{\footrulewidth}{0pt} % Remove footer underlines
\setlength{\headheight}{13.6pt} % Customize the height of the header

\numberwithin{equation}{section} % Number equations within sections (i.e. 1.1, 1.2, 2.1, 2.2 instead of 1, 2, 3, 4)
\numberwithin{figure}{section} % Number figures within sections (i.e. 1.1, 1.2, 2.1, 2.2 instead of 1, 2, 3, 4)
\numberwithin{table}{section} % Number tables within sections (i.e. 1.1, 1.2, 2.1, 2.2 instead of 1, 2, 3, 4)

\setlength\parindent{0pt} % Removes all indentation from paragraphs - comment this line for an assignment with lots of text

%----------------------------------------------------------------------------------------
%	TITLE SECTION
%----------------------------------------------------------------------------------------

\newcommand{\horrule}[1]{\rule{\linewidth}{#1}} % Create horizontal rule command with 1 argument of height

\title{	
\normalfont \normalsize 
\textsc{Department of Computer Science, Rochester Institute of Technology} \\ [25pt] % Your university, school and/or department name(s)
\horrule{2pt} \\[0.4cm] % Thin top horizontal rule
\huge Narrowing the Edge Folkman Number Bounds \\ 
\Large Independent Study Proposal \\
\horrule{2pt} \\[0.5cm] % Thick bottom horizontal rule
}

\author{Christopher Wood \\ Advisor: Professor Stanis{\l}aw Radziszowski} % Your name

\date{\normalsize\today} % Today's date or a custom date

\begin{document}

\maketitle % Print the title

%----------------------------------------------------------------------------------------
%	PROBLEM 1
%----------------------------------------------------------------------------------------

\section{Background}
Edge Folkman numbers, first introduced by Folkman in 1970 \cite{Folkman}, are concerned
with the study of graphs in which a monochromatic coloring of a particular subgraph always exists. 
We write $G \to (a_1, ..., a_k; p)^e$ if for ever edge coloring of an undirected simple graph $G$ not 
containing $K_p$, there exists a monochromatic $K_{a_{i}}$ in color $i$ for some $i \in \{1, ..., k\}$. 
The edge Folkman number is defined as $F_e(a_1, ..., a_k) = \min\{|V(G)| : G \to (a_1, ..., a_k; p)^e\}$.
In 1970 Folkman proved that for all $k > \max(s,t)$, edge- and vertex- Folkman numbers $F_e(s,t;k$
and $F_v(s,t;k)$ exist. Prior to this, Erdos and Hajnal pose the problem of finding $F_e(3,3;4)$, which
can be informally stated as the following \cite{Erdos01}: \\

\noindent \emph{What is the order of the smallest $K_4$-free graph for which any $2$-coloring of its edges must contain at least one 
monochromatic triangle?}\\

This is equivalent to finding the smallest $K_4$-free graph that is not the union of two 
triangle-free graphs. Since the proposition of this problem, there has been a significant
amount of work towards aimed at narrowing the upper and lower bounds of $F_e(3,3;4)$. 
Table \ref{tab:history} enumerates the work on this problem and leads us to the 
current state of the field.

% TOO: insert table from alex's paper here.
\begin{table}
\caption{History of $F_e(e,e;4)$}
\begin{center}
	\begin{tabular}{c|c|l|c}
	\hline
	Year & Bounds & Who & Ref. \\ \hline
	1967 & any? & Erd\H os-Hajnal & \cite{Erdos01} \\
	1970 & exist & Folkman & \cite{Folkman} \\
	1972 & $\geq 10$ & Lin & \cite{lin} \\
	1975 & $\leq 10^{10}$? & Erd\H os offers \$100 for proof & ~ \\
	1986 & $\leq 8 \times 10^{11}$ & Frankl-R\H odl & \cite{frankl86} \\
	1988 & $\leq 3 \times 10^9$ & Spencer & \cite{spencer88} \\
	1999 & $\geq 16$ & Piwakowski et al (implicit) & \cite{piwakowski99} \\
	2007 & $\geq 19$ & Radziszowski-Xu & \cite{spr07} \\
	2008 & $\leq 9697$ & Lu & \cite{lu08} \\
	2008 & $\leq 941$ & Dudek-R\H odel & \cite{dudek08} \\
	2012 & $\leq 786$ & Lange et al & TODO \\
	2012 & $\leq 100$? & Garaham offers \$100 for proof & ~ \\
	\hline
	\end{tabular}
\end{center}
\label{tab:history}
\end{table}

\section{Proposed Work}
The current lower bound for $F_e(3,3;4)$ stands at $19$ (TODO: CITE). A significant step
forward would be to push this bound to $20$ using massive computations. Naturally, it is 
infeasible to enumerate all possible graphs on $20$ vertices and check to see if the arrowing 
property does not hold. 

TODO: lower bound, pushing towards 20 using large-scale computations
TODO: upper bound attacking G127, and subgraph extensions using 3-sat solvers

\section{Outcomes and Deliverables}
Publication-ready paper and more in-depth progress report for the entire project


%%%%% REFERENCES %%%%%

\begin{thebibliography}{9}

\bibitem{Folkman} Jon Folkman. Graphs with monochromatic complete subgraphs in
every edge coloring. \emph{SIAM Journal of Applied Mathematics}. 18 (1970), 19-24.

\bibitem{Erdos01} P. Erd\H os, A. Hajnal. Research problem 2-5. \emph{Journal of Combinatory Theory}, 
2 (1967), 104.

\bibitem{lin} Shen Lin. On Ramsey numbers and $K_r$-coloring of graphs. \emph{Journal of
Combinatiorial Theory, Series B}, 12:82-92, 1972.

\bibitem{frankl86} Peter Frankl and Vojtech R\H odl. Large triangle-free subgraphs
in graphs with $K_4$. \emph{Graphs and Combinatorics}, 2:135-144, 1986.

\bibitem{spencer88} Joel Spencer. Three hunder million points suffice. \emph{Journal of
Combinatiorial Theory, Series A}, 49(2):210-217, 1988. Also see erratum by M. Hovey in
Vol. 50, p. 323.

\bibitem{piwakowski99} Konrad, Piwakoswki, Stanis{\l}aw P. Radziszowski, and Sebastian
Urba\'nski. Computation of the Folkman Number $F_e(3,3;5)$. \emph{Journal of Graph
Theory}, 32:41-49, 1999.

\bibitem{spr07} Stanis{\l}aw P. Radziszowski and Xiaodong Xu. On the Most Wanted
Folkman Graph. \emph{Geocombinatiorics}, 16(4):367-381, 2007.

\bibitem{lu08} Linyuan Lu. Explicit Construction of Samll Folkman Graphs. \emph{SIAM Journal
on Discrete Mathematics}, 21(4):1053-1060, January 2008.

\bibitem{dudek08} Andrzej Dudek and Vojtech R\H odel. On the Folkman Number $f(2,3,4)$.
\emph{Experimental Mathematics}, 17(1):63-67, 2008.

\end{thebibliography}

\end{document}
