\documentclass[paper=a4, fontsize=11pt]{scrartcl} % A4 paper and 11pt font size

\usepackage[T1]{fontenc} % Use 8-bit encoding that has 256 glyphs
\usepackage{fourier} % Use the Adobe Utopia font for the document - comment this line to return to the LaTeX default
\usepackage[english]{babel} % English language/hyphenation
\usepackage{amsmath,amsfonts,amsthm} % Math packages
\usepackage{algorithm}
\usepackage{algorithmicx}
\usepackage{algpseudocode}

\usepackage{lipsum} % Used for inserting dummy 'Lorem ipsum' text into the template

\usepackage{sectsty} % Allows customizing section commands
%\allsectionsfont{\normalfont\scshape} % Make all sections centered, the default font and small caps

\usepackage{color}
\usepackage{fancyhdr} % Custom headers and footers
\pagestyle{fancyplain} % Makes all pages in the document conform to the custom headers and footers
\fancyhead{} % No page header - if you want one, create it in the same way as the footers below
\fancyfoot[L]{} % Empty left footer
\fancyfoot[C]{\thepage} % Empty center footer
\fancyfoot[R]{} % Page numbering for right footer
\renewcommand{\headrulewidth}{0pt} % Remove header underlines
\renewcommand{\footrulewidth}{0pt} % Remove footer underlines

\usepackage[compact]{titlesec}
\titlespacing{\section}{0pt}{*0}{*0}
\titlespacing{\subsection}{0pt}{*0}{*0}
\titlespacing{\subsubsection}{0pt}{*0}{*0}

\setlength{\headheight}{13.6pt} % Customize the height of the header
\setlength{\parskip}{\baselineskip}%
\setlength{\parsep}{0pt}
\setlength{\headsep}{0pt}
\setlength{\topskip}{0pt}
\setlength{\topmargin}{0pt}
\setlength{\topsep}{0pt}
\setlength{\partopsep}{0pt}

%\numberwithin{equation}{section} % Number equations within sections (i.e. 1.1, 1.2, 2.1, 2.2 instead of 1, 2, 3, 4)
%\numberwithin{figure}{section} % Number figures within sections (i.e. 1.1, 1.2, 2.1, 2.2 instead of 1, 2, 3, 4)
%\numberwithin{table}{section} % Number tables within sections (i.e. 1.1, 1.2, 2.1, 2.2 instead of 1, 2, 3, 4)

\setlength\parindent{0pt} % Removes all indentation from paragraphs - comment this line for an assignment with lots of text

%----------------------------------------------------------------------------------------
%	TITLE SECTION
%----------------------------------------------------------------------------------------

\newcommand{\horrule}[1]{\rule{\linewidth}{#1}} % Create horizontal rule command with 1 argument of height
\newcommand{\TODO}{{\color{red}TODO}}

\title{	
\normalfont \normalsize 
\textsc{Department of Computer Science, Rochester Institute of Technology} \\ [25pt] % Your university, school and/or department name(s)
\horrule{2pt} \\[0.4cm] % Thin top horizontal rule
\huge Performance Evaluation of $k-SAT$ Solvers\\ 
\Large Applied to Graph Arrowing \\
\horrule{2pt} \\[0.5cm] % Thick bottom horizontal rule
}

\author{Christopher Wood \\ Advisor: Professor Stanis{\l}aw Radziszowski} % Your name

\date{\normalsize \today} % Today's date or a custom date

\begin{document}

\maketitle % Print the title

\section{Introduction}
In his 1972 seminal paper entitled, ``Reducibility Among Combinatorial Problems,'' Karp introduced a 
list of 21 NP-complete problems, including Boolean satisfiability, the maximum cut of a graph, 
and 0-1 integer programming \cite{karp72}. The complexity of these problems was proven by deriving
a polynomial-time reduction from CIRCUIT-SAT = $\{\langle C \rangle : C$ is a satisfiable Boolean combinational
circuit $\}$, the first problem shown to be NP-complete by Cook in 1971 \cite{cook71-np}, starting
the rush of complexity theory research. 

The problem $3-SAT$, or more formally, $3-CNF SAT$, is a special case of satisfiability.
It is a decision problem in which takes as input a 3-CNF Boolean formula and returns YES if 
the formula is satisfiable, and NO otherwise \cite{clrs90-algorithms}. A 3-CNF formula, more formally known as a Boolean 
formula in 3-conjunctive normal form, is expressed as the Boolean AND of arbitrarily many clauses,
where each clause is the Boolean OR of three literals, which is a Boolean variable or its negation. 
Such a Boolean formula is said to be satisfiable if and only if there exists an assignment of truth 
values to the variables such that substituting them into the literals of the formula will cause it 
to evaluate to true (or 1). Expressed as a formal language, we have that 
$3-SAT = \{\langle \phi \rangle : \phi \text{ is satisfiable }\}$.

In 2002 Hans van Maaren of and John Franco initiated the public SAT competition in
search of optimal performing SAT solvers judged by a variety of criteria and specializations, 
including their ability to demonstrate satisfiability and exhaustively prove unsatisfiability.
In addition, since $SAT$ is a problem that often arises in academia and the industry, 
each of the candidate solvers are rigorously tested with massive application-specific, 
crafted, and random Boolean formulas as input. With three different solver
specializations tested against three different types of inputs, and a first, second, 
and third place awarded to the candidates, a total of $27$ possible trophies are
awarded each year. In the most recent competition held in 2011, solvers were
tested using CPU time and world-clock time as a basis for their results, thus expanding
the trophy space to $54$ slots. The next competition is slated to take place in 2013. 

%TODO: sat problem, sat competition, categories of evaluation, types of problems, graph arrowing...
The $SAT$ problem is particularly interesting when applied to graph arrowing. It can be shown
how to reduce the question $G \overbrace{\to}^\text{?} \to (3,3)^e$ to an equivalent $3-CNF$ formula
$\phi_G$ such that $G \not \to (3,3)^e \Leftrightarrow \phi_G$. Intuitively, this is a very promising
technique for determining if $G \to (3,3)^e$ for $K_4$-free graphs $G$.

The immediate application of this technique is to attack the upper bound of the Folkman number
$F_e(3,3;4)$. In particular, to lower this bound, we will need to find a 
graph $G$ on $n$ vertices where $G \to (3,3)^e$. It has been conjectured that 
$G_{127} = G(127,3) = (\mathbb{Z}_{127}, E = \{(x,y) | x - y = \alpha^3 \mod 127\})$ is a
prime candidate for witnessing an upper bound of $127$ because of its denseness and large number
of triangles. To determine whether $G_{127}$ is indeed a witness we will decompose 
the problem of arrowing into problems on subgraphs $H$ that witness $H \not \to (3,3)^e$, 
and then carefully extend $H$ to encompass all of $G$. For each subgraph $H$ will generate
a corresponding $3$-$CNF$ formula $\phi_G$ by mapping the edges in $E(H)$ to variables 
in $\phi_H \in$ 3-SAT, and for edge adding the following clauses to $\phi_H$:
\begin{align*}
(x + y + z) \wedge (\bar{x} + \bar{y} + \bar{z})
\end{align*}
If $H$ can be extended to encompass all of $G$ and $\phi_G$ is 
shown to be unsatisfiable, then $G \to (3,3)^e$, and so $F_e(3,3;4) = 127$.
Otherwise, if $\phi_G$ is satisfiable, then $G \not \to (3,3)^e$, disproving the
conjecture made in by Radziszowski et al. \cite{Radziszowski07-1}.

Unfortunately, while this approach seems simple at face, the complexity of the formulas
$\phi_H$ with $n \approx 85$ has proven to be very difficult for modern SAT solvers
to handle. In this study, we will attempt to determine the structure of these formulas
that makes them so difficult to solve, and as a result, discover the underlying
cause for the phase transition that occurs at $n \approx 85$. 
We will also present a performance comparison for the popular 
SAT solvers entered into the SAT competition, including Minisat \cite{minisat}, 
zChaff \cite{chaff, zchaff}, and glucose \cite{Audemard09-1}. 

%Minisat, zChaff, glucose, ppfolio //, ppfolio seq, contrasat hack, and 3S.

\section{$SAT$ Solver Algorithms}
The algorithms used in state-of-the-art $SAT$ solvers tend to be based on the famous
DPLL backtracking algorithm proposed by David, Putnam, Logemann, and Loveland in \cite{DPLL}.
The DPLL algorithm works by recursively splitting a CNF formula into smaller formulas
by assigning truth values to individual variables and simplifying the resulting
formula at each level of iteration. Simplification works by removing all clauses in $\phi$ that
become true under the new assignment and all literals which are false. 
This process continues recursively until the formula is deemed satisfiable
or a conflict emerges. Satisfiability occurs when all variables have been assigned truth values 
and the resulting formula has non-empty clauses remaining, or when all clauses can be
evaluted to true. If a conflict arises, the algorithm backtracks to
the last assigned truth value, attempts to substitute the negated truth value, and then proceeds
as normal. A formula $\phi$ is deemed unsatisfiabile if this algorithm exhaustively
checks all possible truth value assignments without yielding a satisfiable formula. 

The DPLL algorithm further enhances the backtracking search through the use of two additional
procedures, unit propagation and pure literal elimination. Unit propagation works by assigning
an appropriate truth value to all unit clauses (i.e. all clauses with only a single literal) and
then propagating this truth value to the rest of the formula. For example, unit propagation
of the formula $\phi = [(x_1) \land (x_1 \lor x_2 \lor x_3)]$ yields $\phi = [(x_2 \lor x_3)]$ by
assigning a value of true to $x_1$. Pure literal elimination works by finding variables
with have uniform polarity (i.e. always positive or negative). For all such literals, there
always a exists a truth assignment such that the containing clauses will evaluate to true.
Therefore, such clauses are delete from $\phi$. For example, given the formula 
$\phi = [(x_1 \lor x_2 \lor \lnot x_3) \land (x_2 \lor \lnot x_3 \lor x_4) \land (x_1 \lor x_2 \lor \lnot x_4)]$,
performing pure literal elimination yields 
$\phi = [(x_1 \lor x_2 \lor \lnot x_4)]$ because we can assign $x_3 = false$ and eliminate
the first two clauses. These two procedures are shown in the pseudocode description shown
in Algorithm \ref{alg:DPLL}.

\begin{algorithm}[t] %[htb]
\caption{DPLL Algorithm \cite{DPLL}} \label{alg:DPLL}
\begin{algorithmic}[1]
\Require{$\phi$}
\Ensure{$true$ or $false$}
\If{$\phi$ is a tautology with the current assignment}
	\Return{$true$}
\EndIf
\If{$\phi$ contains an empty clause or all literals in a clause are false (conflict)}
	\Return{$false$}
\EndIf
\ForAll{unit clauses $c \in \phi$}
	\State{$\phi \gets UnitPropagate(c, \phi)$}
\EndFor
\ForAll{pure literal $l \in \phi$}
	\State{$\phi \gets PureLiteralElimination(l, \phi)$}
\EndFor
\State{$l \gets ChooseLiteral(\phi)$}\\
\Return{$DPLL(\phi \land l) \lor DPLL(\phi \land \lnot l)$}
\end{algorithmic}
\end{algorithm}

\subsection{Advanced Heuristics}
\TODO: print SAT solver papers for reference
\TODO: (4/1/13 evening) conflict analysis, clause recording, boolean constraint propagation, conflict-driven

\section{$3$-$SAT$ Phase Transition}
TODO

\section{Experiments}
In our experiments we denote the number of vertices removed from the graph $G_{127}$ as $m$.
The CNF formula $\phi_G$ for $G_{127}$ experiences a peculiar phase transition when $m \approx 41$,
making it seemingly intractable for modern SAT solvers to halt with an answer in a realistic 
amount of time. Table \ref{tab:performanceSat} shows how the Minisat solver time increases 
as $m$ decreases from $46$ to $43$. The CPU time continues to increase as $m$ decreases.

\begin{table}
	\caption{Performance of Minisat solver by removing $N_R$ vertices from $G_{127}$}
	\begin{tabular}{c | c | c | c | c | c | c}
		\hline
		$N_R$ & CPU Time & Restarts & Conflicts & Decisions & Propagations & Conflict Literals \\ \hline
		46 & 1228.97s & 28668 & 18968226 & 41661327 & 1999748077 & 930184526 \\ 
		45 & 4362.29s & 66558 & 52952393 & 106164381 & 5779728314 & 2755169058 \\ 
		44 & 5576.51s & 81915 & 62420966 & 126120884 & 6748259547 & 3254804365 \\ 
		43 & 21681.5s & 245755 & 199022470 & 384078647 & 21687371442 & 10581492993 \\ 
		\hline
	\end{tabular}
	\label{tab:performanceSat}
\end{table}

To understand the nature of the phase transition, we conducted the experiments
described in the following sections.

\subsection{Experiment 1: Subgraph Reduction and Variable Assignment Propagation}
Breaking the problem of $G_{127} \to (3,3)^e$ into small graphs $H$ is a natural way
to approach this problem. In addition, clause reduction and unbalancing through
variable assignment may provider further simplifications to $\phi_H$.
Therefore, for this experiment, we have the following parameters:
\begin{itemize}
	\item $m$ - the number of vertices in $V \subset V(G)$ such that $H = G[V]$. The selection of
	these vertices will be both structured and unstructured. That is, we will experiment
	with removing structured groups of vertices, such as those contained within independent 
	sets, as well as unstructure groups composed of randomly selected vertices.
	\item $r$ - the number of variables $x_i \in \phi_H$ assigned truth values that are
	propogated through the rest of the formula. 
	\item $p$ - the number of levels of recursive propagation that occur after assigning 
	$r$ truth values.
\end{itemize}
Clause reduction will use the same unit propagation and pure literal elimination techniques
presented in the DPLL algorithm as a preprocessing step for $\phi_H$ before it is run with a solver. 

\TODO: run the experiments and drop data here for DIFFERENT solvers

\begin{thebibliography}{9}

\bibitem{karp72} R. Karp, Reducibility among combinatorial problems, \emph{Complexity of Computer Computations, (RE Miller and JM Thatcher, eds.)} (1972), 85-103.

\bibitem{cook71-np} S. A. Cook, The Complexity of Theorem-Proving Procedures, \emph{In Proceedings of the third annual ACM symposium on Theory of computing (STOC '71)}. ACM, New York, NY, USA (1971), 151-158. DOI=10.1145/800157.805047. {\tt http://doi.acm.org/10.1145/800157.805047}

\bibitem{clrs90-algorithms} T. H. Gormen, C. E. Leiserson, R. L. Rivest, C. Stein, Introduction to Algorithms, \emph{MIT Press} \textbf{44} (1990), 97-138.

\bibitem{Radziszowski07-1} S. P. Radziszowski, X. Xu, On the Most Wanted Folkman Graph, \emph{Geocombinatiorics} \textbf{16} (4) (2007), 367-381.

\bibitem{minisat} N. S\"{o}rensson, N. E\`{e}n, Minisat v1.13 - A SAT Solver with Conflict-Clause Minimization, \emph{SAT 2005} (2005) 53.

\bibitem{zchaff} Y. Mahajan, Z. Fu, S. Malik, Zchaff2004: An Efficient SAT Solver, \emph{Theory and Applications of Satisfiability Testing}. Springer Berlin/Heidelberg (2005).

\bibitem{chaff} M. W. Moskewicz, C. F. Madigan, Y. Zhao, L. Zhang, S. Malik, Chaff: Engineering an efficient SAT solver, \emph{Proceedings of the 38th annual Design Automation Conference} ACM (2001).

\bibitem{Audemard09-1} G. Audemard, L. Simon, GLUCOSE: a solver that predicts learnt clauses quality, \emph{SAT Competition} (2009), 7-8.

\bibitem{DPLL} M. Davis, G. Logemann, and D. W. Loveland, A machine program for theorem-proving, \emph{Communications of the ACM} \textbf{5(7)} (1962), 394-397.

\end{thebibliography}

\end{document}